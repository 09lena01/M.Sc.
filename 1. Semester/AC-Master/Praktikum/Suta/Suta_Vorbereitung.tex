\documentclass[11pt]{scrreport}
\usepackage[utf8]{inputenc}
\usepackage{lmodern}
\usepackage[german] {babel}
\usepackage[T1]{fontenc}
\usepackage{graphicx,float}
\usepackage{booktabs}
\usepackage{amsfonts}
\usepackage{amsmath}
\usepackage[dvipsnames]{xcolor}
\usepackage{suffix}
\usepackage{xstring}
\usepackage[onehalfspacing]{setspace}
\usepackage{chemformula}
\usepackage[breaklinks,pdfpagelabels,hidelinks]{hyperref}%
\usepackage[figure]{hypcap}
\usepackage{cleveref}
\usepackage[style=chem-angew, backend=biber,doi,chaptertitle,articletitle,abbreviate=false]{biblatex}
\usepackage{chemfig}
\usepackage[right=3.5cm, left=2.5cm, top=2.5cm, bottom=2.5cm]{geometry}%,showframe
\usepackage{ragged2e}
\usepackage{caption}
\usepackage[section]{placeins}
\usepackage{siunitx}
\sisetup{detect-all,locale=DE}


\raggedbottom
\begin{document}\noindent
	\chapter{Theorie}
	\textbf{Übersättigung}	\\ 
	Übersättigung tritt ein, wenn die Konzentrationen der Metallkationen und des Reduktionsmittels in Lösung sehr hoch sind. Dies führt zur Bildung vieler kleiner Nanopartikel einheitlicher Größe.\\
	\textbf{Bulk}	\\ 
	Bezeichnet das Volumen eines Festkörpers, in dem möglichst viele Atome koordinativ abgesättigt werden können. Der Begriff wird auch verwendet, um makroskopische metallische Festkörper zu bezeichnen, deren optische Eigenschaften von denen nanoskaliger Partikel abweichen.\\
	\textbf{Ostwald-Reifung}	\\ 
	Ein typischerweise diffusionskontrollierter Vorgang des Partikelwachstums. Aufgrund des hohen Drucks an der Grenzfläche (bedingt durch Oberflächenspannung) haben kleinere Partikel eine erhöhte Löslichkeit, da ihre ungesättigten Oberflächenatome durch Lösungsmittelmoleküle koordiniert werden können. Entsprechend dem ersten Fick’schen Gesetz lösen sich kleine Nanokristalle auf (hohe Konzentration), und die Atome wandern zu den bereits größeren Kristallen (kleinere Konzentration), wodurch diese wachsen.\\
	\textbf{monodispers}	\\ 
	Beschreibt Partikel, die eine einheitliche Größe aufweisen. Die Ostwald-Reifung kann genutzt werden, um relativ monodisperse Partikel einer gewünschten Größe herzustellen.\\
	\textbf{LaMer-Dinegar-Modell}	\\ 
	ie zugrundeliegende Modellvorstellung, die das Partikelwachstum in Lösung beschreibt. Dieses Modell stellt den schematischen zeitlichen Verlauf der Konzentration der gelösten Metallkationen bei der Bildung kleiner Nanokolloide dar. Durch geschickte Wahl der Konzentrationen, Reaktionszeit und Temperatur kann die Größe der gewünschten Nanokolloide zuverlässig beeinflusst werden.\\
	\textbf{Quantenpunkte, 2023 Nobelpreis Chemie}	\\ 
	Quantenpunkte sind halbleitende Nanokristalle (wie CdCh), deren Größe genau kontrolliert wird, um ihre Eigenschaften zu steuern. Die kontrollierte Herstellung dieser Partikel nach dem LaMer-Dinegar-Modell brachte Moungi Bawendi, Louis Brus und Aleksei Ekimov 2023 den Nobelpreis für Chemie ein. Je nach Größe lumineszieren die Kristallite in unterschiedlichen Farben.
	\\
	\textbf{Agglomeration}	\\
	Der Prozess, bei dem kleine Nanopartikel zu größeren Partikeln zusammenwachsen. Die chemische Stabilisierung von Nanopartikeln zielt darauf ab, die Agglomeration zu verhindern.
	\newpage\noindent
	\textbf{Ausflockung}	\\ 
	Ein Prozess, der ebenso wie die Agglomeration sterisch oder elektronisch verhindert werden muss, um Nanopartikel in Dispersionen stabil zu halten.
	\\
	\textbf{Hülle}	\\ 
	Eine schützende Schicht, die von organischen Verbindungen gebildet wird, die sich an der Oberfläche von Nanopartikeln anordnen, um deren Agglomeration und Auflösung zu verhindern.
	\\
	\textbf{Ligand}	\\ 
	Organische Moleküle, die einer Dispersion zugesetzt werden, sich an der Oberfläche der Nanopartikel anordnen und eine schützende Hülle bilden, um die Partikel zu stabilisieren.
	\\
	\textbf{Kolloid}	\\ 
	Ein insgesamt funktionalisiertes Partikel, das aus dem Nanopartikel und der schützenden Hülle aus Liganden besteht.
	\\
	\textbf{Dispersion}	\\ 
	Das System, das die Kolloide und das Dispersionsmittel (Lösungsmittel) umfasst.
	\\
	\textbf{Sol}	\\ 
	Die Bezeichnung für die gesamte Dispersion (System Kolloide + Dispersionsmittel). Die Farbigkeit solcher Sole (Nanosole) wird typischerweise mittels Absorptionsspektroskopie untersucht.
	\\
	\textbf{Drude-Modell}	\\ 
	Eine Modellvorstellung zur Beschreibung metallischer Festkörper. Es betrachtet Metalle als eine Ansammlung fixierter Metallkationen-Rümpfe in einem „See“ (Gas) aus frei beweglichen, delokalisierten Elektronen.
	\\
	\textbf{Feldkomponente $E$}	\\ 
	Die elektrische Feldkomponente des Lichts (einer elektromagnetischen Welle), die durch die wirkende Lorentzkraft Elektronen verschiebt und die beweglichen Elektronen des Metalls zur Schwingung um ihre Gleichgewichtslage anregt.
	\\
	\textbf{elektrisches Dipolmoment $\mu_{el}$}	\\ 
	Wird durch die Verschiebung r eines Elektrons aus der Gleichgewichtslage erzeugt. Es hat die Einheit Cm und ist definiert als $\mu_{el}=-er$.
	\\
	\textbf{Polarisation $P$}	\\ 
	Eine makroskopische Größe (Einheit:$Cm^2$) im Metall, die durch die Vielzahl der erzeugten Dipolmomente entsteht. Sie ist im Mittel die Summe aller durch das Licht induzierten Dipolmomente, normiert auf ein Volumen ($P=n_e\mu_{el}$) und ist direkt proportional zur einwirkenden Feldstärke E. In kleineren Partikeln ist die Polarisation geometrisch bedingt kleiner.
	\\
	\textbf{relative elektrische Permittivität ${\varepsilon}_r$}	\\ 
	Eine dimensionslose Größe (${\varepsilon}_r=1+\chi_{el}$) die sich aus der elektrischen Suszeptibilität $\chi_{el}$ ergibt und eng mit den optischen Eigenschaften von Festkörpern verknüpft ist. Sie ist von der Frequenz $\omega$ des eingestrahlten Lichts abhängig.
	\newpage\noindent
	\textbf{Plasmafrequenz}	\\ 
	Die natürliche Eigenfrequenz der kollektiv schwingenden Elektronen im Metall ($\omega_p=\sqrt{\frac{n_ee^2}{\varepsilon_0m_e}}$). Sie hängt von der Elektronendichte $n_e$ des Metalls ab. Bei Lichtfrequenzen unterhalb der Plasmafrequenz reflektiert ein Metall total, bei Frequenzen oberhalb wird es transparent.
	\\
	\textbf{Plasma}	\\ 	
	Der ionisierte Zustand eines Gases, in dem frei bewegliche Ladungsträger vorkommen. Es wird oft als vierter Aggregatzustand bezeichnet und entsteht bei hohen Temperaturen, die zur Ionisierung von Atomen ausreichen. Das Drude-Modell idealisiert das Metall als Plasma.
	\\
	\textbf{elektrisches Feld $E_{lokal}$}	\\ 
	Das elektrische Feld, das ein metallisches Nanopartikel als oszillierender Dipol ausstrahlt, mit der gleichen Frequenz wie das eingehende Feld. Die Oberflächen-Plasmawelle verstärkt dieses lokal vorliegende elektrische Feld auf der Partikeloberfläche.
	\\
	\textbf{Rayleigh-Streuung}	\\
	Elastische Streuung von eingehendem sichtbarem Licht durch Nanopartikel, die sehr viel kleiner sind als die Wellenlänge des Lichts.
	\\
	\textbf{nanoskalige Antennen}	\\
	Metallische Nanopartikel können als nanoskalige Antennen dienen, da die Oberflächen-Plasmawelle das lokal vorliegende elektrische Feld auf der Oberfläche verstärkt, wodurch beispielsweise die Lumineszenz von dort angebrachten Molekülen enorm verstärkt werden kann.
	\\
	\textbf{stehende Welle}	\\ 
	Eine stabile Oberflächen-Plasmawelle, die sich in einem geometrisch eingeschränkten Nanopartikel ausbildet und eine Quantisierungsbedingung zur Folge hat.
	\\
	\textbf{Quantisierungsbedingung}	\\ 
	Eine Bedingung, die durch die Geometrie des Nanopartikels vorgegeben ist. Sie führt dazu, dass eine stehende Oberflächen-Plasmawelle nur durch Licht einer bestimmten resonanten Frequenz gebildet werden kann.
	\\
	\textbf{lokalisiertes Oberflächenplasmon/Partikelplasmon}	\\ 
	Eine Anregung, bei der ein resonantes Photon in ein Teilchen umgewandelt wird, das einer resonanten Plasmawelle entspricht. Diese Anregung wird als lokalisiertes Oberflächenplasmon (LSPR) oder passender als Partikelplasmon bezeichnet. Diese Plasmonen sind an der Oberfläche der Nanopartikel lokalisiert.
	\\
	\textbf{Clausius-Mossotti-Gleichung}	\\ 
	Eine Gleichung, die verwendet werden kann, um die mikroskopische Polarisierbarkeit $\alpha$ eines sphärischen Nanopartikels mit Radius $R$ in einem Dispersionsmedium zu bestimmen:$\alpha=4\pi R^3\frac{\varepsilon_m-\varepsilon_d}{\varepsilon_m+2\varepsilon_d}$\\
	\textbf{photonischer Effekt}	\\
	Ein Effekt, bei dem die lokalisierte Partikelplasmonenresonanz durch das umgebende Lösungsmittel verschoben wird (auch als nano-photonischer Effekt bezeichnet), da das Lösungsmittel durch seine eigene Polarität polarisierend auf die Nanopartikel wirkt.
	\\ 
	\textbf{Komplementärfarbe}	\\ 
	Die Farbe, die das menschliche Auge registriert. Da ein Stoff nur Licht bestimmter Wellenlängen absorbiert, sieht man immer die Farbe, die der Absorptionswellenlänge komplementär ist, da sie vom restlichen reflektierten Teil des Weißlichts hervorgerufen wird.
	\\
	\textbf{Absorptionsspektroskopie}	\\
	ine Untersuchungsmethode, mit der die Farbigkeit und die Absorptionswellenlängen von Stoffen quantitativ erfasst werden können. Sie dient zur Untersuchung der Farbigkeit von Solen.
	\\
	\textbf{Transmission}	\\ 
	Die Messung von Absorptionsspektren, die typischerweise für transparente Lösungen durchgeführt wird (lat. transmittere: übersenden).
	\\
	\textbf{Blazegitter}	\\ 
	Wird im Monochromator moderner UV-Vis-Absorptionsspektrometer verwendet (oft auch als Stufengitter bezeichnet), um das Weißlicht einer Lampe in seine einzelnen Spektralanteile zu zerlegen.
	\\
	\textbf{Transmissionsgrad $T({\lambda})$}	\\ 
	Definiert als das Verhältnis zwischen der abgeschwächten Intensität $I({\lambda})$ nach Durchgang durch die Probe und der Referenzintensität $I_0$: $T(\lambda)=\frac{I(\lambda)}{I_0}$.\\
	\textbf{Extinktion $E({\lambda})$/optische Dichte $OD({\lambda})$}	\\ 
	Eine Größe, die mit dem Transmissionsgrad verknüpft ist: $E(\lambda)=OD(\lambda)=-\log T(\lambda)$. Sie dient als Maß für die Absorption einer bestimmten Wellenlänge durch die Lösung bei einer definierten Schichtdicke $d$.\\
	\textbf{Absorptionskoeffizient ${\alpha}({\lambda})$}\\
	Ein Maß für die Absorption, das vom Einfluss der Schichtdicke $d$ unabhängig ist. Er ist definiert als $\alpha(\lambda)=\frac{\ln 10}{d}E(\lambda)$ und wird typischerweise in Einheiten von $cm^{-1}$ angegeben. Der Kehrwert des Absorptionskoeffizienten gibt die Schichtdicke an, bei der die Lichtintensität auf etwa $37\,\%$ des Ausgangswertes abgeklungen ist.
	\newpage
	\chapter{Durchführung}
	\section{Herstellung stabilisierter Ag-Nanokeime}
	\begin{itemize}
		\item $5\,mL$ $2,5\,mM$ Trinatriumcitrat (Dihydrat; \ch{C6H5Na3O7* 2 H2O}) herstellen
		\item[=>] $n=2,5\,\frac{mmol}{L}*{\color{red}5\cdot 10^{-3}\,L}=0,0125\,mmol=1,25\cdot 10^{-5}\,mol;$
		\item[=>] $m=1,25\cdot 10^{-5}\,mol*294,1\,\frac{g}{mol}=3,68\cdot 10^{-3}\,g={\color{red}3,68\, mg}$
		\item Unter rühren: Zugabe $4\,mL$ $0,7\,mM$ Polyvinylpyrrolidon (PVP) + $0,4-0,5\,mL$ $10\,mM$ \ch{NaBH4}
		\item[=>] PVP: $n=0,7\,\frac{mmol}{L}*{\color{red}5\cdot 10^{-3}\,L}=3,5\cdot 10^{-3}\, mmol=3,5\cdot 10^{-6}\, mol$;\\k $m=3,5\cdot 10^{-6}\, mol*40000\,\frac{g}{mol}=0,14\, g={\color{red}140\, mg}$
		\item [=>] \ch{NaBH4}: $n=10\,\frac{mmol}{L}*{\color{red}1\cdot 10^{-3}\,L}=0,01\, mmol=1\cdot 10^{-5}\, mol$;\\ $m=1\cdot 10^{-5}\, mol*37,83\,\frac{g}{mol}=3,78\cdot 10^{-4}\, g={\color{red}0,378\, mg}$
		\item Zugabe $5\,mL$ $0,5\,mM$ \ch{AgNO3} zügig mit einem Schuss unter kontinuierlichem rühren
		\item[=>] $n=0,5\cdot 10^{-3}\,\frac{mol}{L}*{\color{red}10?\cdot 10^{-3}\,L}=5\cdot 10^{-6}\, mol; \\ m=5\cdot 10^{-6}\, mol*169,87\,\frac{g}{mol}=8,49\cdot 10^{-4}\, g={\color{red}0,849\, mg}$
		\item Rühren bei RT für ca. 20 min bis intensive Gelbfärbung
		\item wenn nach 30 min keine Färbung: Zugabe 1-3 Tropfen \ch{NaBH4}-Lösung
		\item Schnappi mit Alufolie umwickeln
		\item UV-Vis-Absorptionsspektrum
	\end{itemize}
	\section{Herstellung stabilisierter Ag-Nanoprismen variabler Größe}
	\begin{itemize}
		\item in Schnappi: 5\,mL bidest. Wasser, 0,1\,mL 10\,mM L-(+)-Ascorbinsäure, variiertes Volumen Ag-Nanokeimlösung, kurz schütteln
		\item[=>] $n=1\cdot 10^{-2}\,\frac{mol}{L}*{\color{red}5\cdot 10^{-4}\,L}=5\cdot 10^{-6}\,mol; \\ m=5\cdot 10^{-6}\,mol*176,12\,\frac{g}{mol}=8,81\cdot 10^{-4}\, g={\color{red}0,881\,mg}$
	\end{itemize}
	\begin{tabular}{|c|c|} \hline
		Laufnummer & Volumen Keimlösung [mL] \\ \hline
		1 & 0,65\\ \hline
		2 & 0,40\\ \hline
		3 & 0,20\\ \hline
		4 & 0,10\\ \hline
		5 & 0,02\\ \hline
	\end{tabular}
	\begin{itemize}
		\item Zugabe 3\,mL 0,5\,mM \ch{AgNO3} mit Geschwindigkeit von etwa $1\,\frac{mL}{min}$ => Farbänderung
		\item Zugabe 0,5\,mL 25\,mM? Trinatriumcitrat-Lösung, stehen lassen für 15 min
		\item[=>] $n=25\cdot10^{-3}\,\frac{mol}{L}*{\color{red}1\cdot10^{-3}\,L}=2,5\cdot10^{-5}\,mol;\\ m=2,5\cdot10^{-5}\,mol*294,1\,\frac{g}{mol}=7,35\cdot10^{-3}\,g={\color{red}7,35\,mg}$
		\item Lösungen (ohne Blitz) nebeneinander aufgestellt fotografieren
		\item UV-Vis-Absorptionsspektrum
	\end{itemize}
	\section{Einfluss des Lösungsmittels auf die Farbe der Ag-Nanokolloide}
	\begin{itemize}
		\item Entnahme einiger Tropfen aus Sol 1 o. 2 (aufschreiben!), Zugabe in Einmalküvette
		\item Auffüllen mit DMSO, schütteln bis zur Homogenisierung
		\item Foto, UV-Vis-Absorptionsspektrum
	\end{itemize}
	\section{Thermodynamisch und kinetisch kontrollierte Formveränderung der Nanoprismen}
	\begin{itemize}
		\item Sol 4 in Schnappi \& Küvette
		\item \textbf{Thermodynamisch}: mit Parafilm verschließen, tauchen in $80-100^{\circ}C$ Wasserbad
		\item Entnahme Teil nach 15, 30, 45, 60 min, Überführen in Küvette: Foto, UV-Vis-Absorptionsspektrum
		\item \textbf{Kinetisch}: Küvette mit UV-Lampe ($\lambda=365\,nm$) bestrahlen, nach 15, 30, 45, 60 min: Foto, UV-Vis-Absorptionsspektrum
		\end{itemize}
\end{document}