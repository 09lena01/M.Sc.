\documentclass[float=false, crop=true]{standalone}
\usepackage[english]{babel}
\usepackage{graphicx}
\usepackage{textgreek}
\usepackage[super, square]{natbib}
\usepackage{chemformula}%Summenformeln
\usepackage[utf8]{inputenc}
\usepackage{ragged2e}
\usepackage{import}
\usepackage{pgfplots}
\usepackage{tikz}
\usepackage{float}
\usepackage{amsmath,amsfonts,amssymb,amsthm,gensymb,textcomp}
\usepackage[makeroom]{cancel}
\usetikzlibrary{calc}

\pgfplotsset{compat=1.17}
\pgfkeys{/pgf/number format/.cd,
	use comma,
	set thousands separator={}}

\begin{document}
	
	\begin{tikzpicture}
		\begin{axis}[
			width=\textwidth,
			height=10cm,
			legend pos=north east,
			legend style={at={(0.97,0.8)},anchor=north east},
			xlabel = {Beugungswinkel $2\theta$ / \degree},
			ylabel = {normierte Intensität},
			%x tick label style={/pgf/number format/.cd, use comma,set thousands separator={}},
			%y tick label style={/pgf/number format/.cd, use comma, set thousands separator={}},
			%yticklabel=\empty,
			xmin = 4, 
			xmax = 51,
			ymin = -1.05,
			ymax = 1.05,
			% scaled x ticks=base 10:3,
			% scaled y ticks=base 10:3,
			% grid=both,
			% xtick = {0,10,20,...,100},
			minor x tick num=4,
			% ytick = {0,0.10,0.20,...,1.00},
			% minor y tick num=4
			% x dir=reverse,
			% y dir=reverse
			title=Pulverdiffraktogramme von MIL-101
			]
			\addplot [red] table [x index=0,y index=1,col sep=semicolon]{C:/Users/assma/OneDrive/Desktop/M._Sc._Chemie/1. Semester/AC-Master/Praktikum/Janiak/MIL-101Cr_Simulation_norm.txt};
			\addplot [black] table [x index=0,y index=1,col sep=semicolon]{C:/Users/assma/OneDrive/Desktop/M._Sc._Chemie/1. Semester/AC-Master/Praktikum/Janiak/MIL101_STD_Gr_15_16_norm.txt};
			\legend{Simulation, mit \ch{HNO3}}
		\end{axis}
	\end{tikzpicture}
	
\end{document}